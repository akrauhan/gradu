%\usepackage[leqno]{amsmath}
%\usepackage{amsthm}
\usepackage{amssymb}
%
\setlength{\textheight}{9in} % amsbook default: 632pt
\setlength{\textwidth}{6in}  % amsbook default: 360pt
\setlength{\topmargin}{0in}
\setlength{\oddsidemargin}{0.4cm}
\setlength{\evensidemargin}{0.4cm}
%
\usepackage[T1]{fontenc}

\usepackage{ae}
\usepackage{bbm}
\usepackage[english,finnish]{babel}
\usepackage[dvipsnames]{xcolor}
\usepackage{transparent}
\usepackage[colorlinks,linkcolor=blue,citecolor=blue,urlcolor=blue,bookmarks=false,hypertexnames=true]{hyperref} 

%% Input encoding: valitse tekstieditorisi käyttämä
%% vaihda my\"os dokumentin ensimmäinen rivi vastaavasti
%\usepackage[utf8]{inputenc}           % !TeX encoding = utf8
%\usepackage[latin1]{inputenc}        % !TeX encoding = latin1
%\usepackage[applemac]{inputenc}    % !TeX encoding = appleroman
%
%\usepackage[dvips]{graphicx}
%\usepackage[pdftex]{graphicx}
\usepackage{graphicx}

\usepackage{subfiles}

\graphicspath{graphics}

\renewcommand\thesection{\arabic{chapter}.\arabic{section}}
\renewcommand\thesubsection{\arabic{chapter}.\arabic{section}.\arabic{subsection}}
\renewcommand\thesubsubsection{\arabic{chapter}.\arabic{section}.\arabic{subsection}.\arabic{subsubsection}}

\renewcommand\thefigure{\arabic{section}}
%
%% AMS-LaTeX -määrityksiä
%
\theoremstyle{plain}
\newtheorem{theorem}{Lause}[chapter]
\newtheorem{lemma}[theorem]{Lemma}
\newtheorem{corollary}[theorem]{Seuraus}
%
\theoremstyle{definition}
\newtheorem{definition}[theorem]{Määritelmä}
\newtheorem{example}[theorem]{Esimerkki}
%
\theoremstyle{remark}
\newtheorem{remark}[theorem]{Huomautus}
%
\numberwithin{equation}{chapter}
\numberwithin{figure}{chapter}
%
%% Uusia komentoja, macroja
%

\newcommand{\R}{\mathbb{R}}
\newcommand{\Rp}{{\mathbb{R}^{+}}}
\newcommand{\N}{\mathbb{N}}
\newcommand{\Q}{\mathbb{Q}}
\newcommand{\Qp}{\mathbb{Q}^{+}}
\newcommand{\E}{\mathcal{E}}
\newcommand{\dt}{\text{d}t}
\newcommand{\Borel}{\mathcal{B}}
\newcommand{\indfct}[1]{\mathbbm{1}_{#1}}

\newcommand{\what}[1]{\colorbox{Lavender}{#1}}            % Selvitä, mitä tarkoittaa tai miksi näin.
\newcommand{\whytho}[1]{\colorbox{orange}{#1}}                   % Perusteltava tarkemmin.
\newcommand{\todo}[1]{\colorbox{SkyBlue}{\textbf{TODO: #1}}}    % Tee myöhemmin tähän.
\newcommand{\define}[1]{\colorbox{YellowGreen}{#1}}              % Käsitteen määritelmä puuttuu.

\DeclareMathOperator*{\esssup}{ess\, sup}
\DeclareMathOperator{\supp}{supp}
\DeclareMathOperator{\diam}{diam}
%
%% Uusia komentoja, päällekirjoitetut
%
\renewcommand{\P}{\mathbf{P}}
\renewcommand{\d}{\text{d}}
\renewcommand{\check}[1]{\colorbox{Lavender}{#1}}              % Varmista, onko oikein.
