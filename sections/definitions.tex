\begin{document}

% Määritelmiä ja lauseita (luonnos)

\subsection{Hallussa:}
\begin{definition}
    konveksi joukko
\end{definition}
\begin{definition}
    etäisyys
\end{definition}
\begin{definition}
    push-forward: $f_\#\mu$
\end{definition}

\begin{definition}
    todennäköisyysmitta
\end{definition}

\subsection{Hataralla pohjalla:}
\begin{definition}
L-ääretön, L-infinity:
    \[L^\infty (A) = \{f:A\to X: ||f||_\infty < \infty\}\] 
    \[||f||_\infty = \sup_{x\in A} ||f(x)||_{\R^n}\]
\end{definition}
\begin{definition}
    1-Lipschitz-kuvaus
\end{definition}


\begin{definition}
    Borelin $\sigma$-algebra
\end{definition}

\begin{definition}
    metrinen avaruus
\end{definition}

\begin{definition}
    todennäköisyysavaruus
\end{definition}

\begin{theorem}
    Skorokhodin lause: Olkoon (K, d) \what{esikompakti} metrinen avaruus ja $\mu$ todennäköisyysmitta joukossa $K$. Tällöin on olemassa parametrisaatio $\chi$ mitalle $\mu$.
\end{theorem}
\begin{proof}
Bernot, Caselles: Optimal transportation networks s.185
\end{proof}

\end{document}