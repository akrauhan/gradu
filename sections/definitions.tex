\begin{document}

% Määritelmiä ja lauseita (luonnos)

\subsection{HALLUSSA:}
\begin{definition}
    \textbf{konveksi joukko:} Olkoon $S$ vektoriavaruus. Joukko $C \subset S$ on \textit{konveksi}, jos kaikille $x, y \in C$ pätee $l(x,y) \subset C$, kun merkitään $l(x,y)$ janan pisteitä pisteestä $x$ pisteeseen $y$.
\end{definition}
\begin{definition}
    \textbf{etäisyys:} Joukon $X$ funktio $d: X\times X \to \R^+$ on \textit{etäisyys}, jos kaikilla $x, y, z \in X$ pätee
    \begin{enumerate}
        \item $d(x,y) \ge 0$
        \item $d(x,y) = 0 \iff x = y$
        \item $d(x,y) = d(y,x)$
        \item $d(x,y) \ge d(x,z) + d(z,y)$
    \end{enumerate}
\end{definition}

\begin{definition}
    \textbf{Sigma-algebra. } Olkoon $X$ joukko. Tällöin $\Gamma \subset \mathcal P(X)$ on \textit{sigma-algebra}, joukossa $X$, jos $\Gamma$ toteuttaa seuraavat ominaisuudet:
    \begin{enumerate}
        \item $\emptyset \in \Gamma$ 
        \item jos $A \in \Gamma$, niin $A^c \in \Gamma$
        \item jos $A_1, A_2, ... \in \Gamma$, niin $\bigcup_{j=1}^\infty A_j \in \Gamma$ 
    \end{enumerate}
    (Lehrbäck, MII, s. 86)
\end{definition}

\begin{definition}
    \textbf{Joukkoperheen virittämä sigma-algebra. } Olkoon $X$ joukko ja olkoon $\Delta \subset \mathcal P (X)$. Tällöin 
        $$ \Gamma_\Delta = \bigcap \{\Gamma : \Gamma \text{ on sigma-algebra joukossa } X \text{ ja }  \Delta \subset \Gamma \}$$ 
    on joukkoperheen $\Delta$ \textit{virittämä} sigma-algebra joukossa $X$.
    (Lehrbäck, MII, s. 86)
\end{definition}

\begin{definition}
    \textbf{Borelin joukko:}
    Olkoon
    \[\Delta = \{A \subset \R^n : A \text{ on avoin joukko}\} \subset \mathcal{P}(\R^n) \]
    Tällöin $\sigma$-algebra $\mathcal B = \mathcal B_n := \Gamma_\Delta$ on avaruuden $\R^n$ \textit{Borelin} $\sigma$-\textit{algebra} ja joukkoja $A\in \mathcal B$ kutsutaan \textit{Borel-joukoiksi.} 
    (Lehrbäck, MII, s.87)
\end{definition}

Sigma-algebran $\Gamma_\Delta$ määritelmän nojalla Borelin joukko on \textit{suppein} avoimista joukoista koostuva joukon $X$ sigma-algebra, joka sisältää joukkoperheen $\Delta$.



\begin{definition}
    \textbf{Borelin avaruus:} Olkoon $X\subset \R^n$ joukko ja $\mathcal A$ sigma-algebra joukossa $X$. Tällöin pari $(X, A)$ on \textit{Borelin avaruus}. Borelin avaruudesta käytetään myös nimitystä \textit{mitallinen avaruus}. LÄHDE?
\end{definition}

\begin{definition}
    \textbf{Push-forward:} Olkoon Borelin avaruudet $(X_1, \Sigma_1)$ ja $(X_2, \Sigma_2)$, mitallinen funktio $f:X_1 \to X_2$ ja mitta $\mu: \Sigma_1 \to \R^+$. Mitan $\mu$ \textit{pusku} on mitta $f_\#\mu: \Sigma_2 \to \R^+$, merkitään
    $$f_\# \mu (B) = \mu(f^{-1}(B)) \text{ kaikilla } B\in \Sigma_2.$$
\end{definition}

\begin{definition}
    \textbf{Alhaalta puolijatkuvuus:} Olkoon $X\subset \R^n$. Funktio $f: X \to \R \cup\{-\infty, \infty\}$ on \textit{alhaalta puolijatkuva} pisteessä $x_0$ jos 
    $$\liminf_{x\to x_0}  f(x) \ge f(x_0).$$
\end{definition}

\begin{definition}
    \textbf{1-Lipschitz-kuvaus:} Olkoon metriset avaruudet $(X, d_x)$ ja $(Y, d_y)$. Funktio $f:X\to Y$ on \textit{Lipschitz-jatkuva}, jos on olemassa $K\ge 0$ s.e. kaikille $x_1,x_2 \in X$
    $$d_y(f(x_1),f(x_2)) \le Kd_x(x_1,x_2).$$
    Funktio on \textit{1-Lipschitz}, jos funktio on Lipschitz-jatkuva vakiolla $K=1$.
\end{definition}

\begin{definition}
    \textbf{Mitta} Oletetaan, että $X$ on joukko ja $\Gamma$ on $\sigma$-algebra joukossa $X$. Funktio $\mu : \Gamma \to [0, \infty]$ on \textit{mitta} (joukossa X tai $\sigma$-algebrassa $\Gamma$), jos 
    \begin{enumerate}
        \item $\mu(\emptyset) = 0$, \\
        ja
        \item Jos $A_1, A_2, ... \in \Gamma$ ovat erillisiä eli $A_i \cap A_j = \emptyset$ aina, kun $i \ne i$, niin
        $$\mu\left(\bigcup_{j=1}^\infty A_j \right) = \sum_{j=1}^\infty \mu(A_j).$$
    \end{enumerate}
    
    Kolmikkoa $(X,\Gamma, \mu)$ sanotaan \textit{mitta-avaruudeksi} ja joukkoja $A\in \Gamma$ sanotaan $\Gamma$-mitallisiksi.
    
    (Lehrbäck, MII, s. 88)
\end{definition}

\begin{definition}
    \textbf{Funktion mitallisuus:} Olkoon $X$ joukko ja $\Gamma \subset \mathcal P (X)$ sigma-algebra. Olkoon $A\in \Gamma$. Funktio $f: A \to \R \bigcup\{-\infty, \infty\}$ on \textit{$\Gamma$-mitallinen}, jos kaikille $a \in \R$ pätee, että
    \begin{equation*}
        \{x \in A : f(x) > a\} = f^{-1}(]a, \infty]) \in \Gamma.
    \end{equation*} (Lehrbäck MII, s. 110)
\end{definition}

\begin{definition}
    \textbf{Lebesguen ulkomitta:} Funktio $m^* : \R^n \to \R^+$ on \textit{Lebesguen ulkomitta} kun määritellään
    $$m^*(A) = \inf\left\{\sum_{i=1}^\infty v(I_i) : I_i \in \mathcal K_n \text{ ja } A \subset \bigcup_{i=1}^{\infty}I_i\right\} \text{ kaikille } A \subset \R^n$$
\end{definition}

\begin{definition}
    \textbf{Lebesgue-mitalliset joukot:} Joukko $A\subset \R^n$ on \textit{Lebesgue-mitallinen}, jos kaikille $E\subset \R^n$ pätee
    $$m^*(E) = m^*(E \, \backslash \, A) + m^*(E\cap A)$$
    Kaikkien avaruuden $\R^n$ Lebesgue-mitallisten joukkojen kokoelmaa merkitään 
    $$\mathcal M = \mathcal M_n = \left\{A\subset \R^n \colon A \text{ on Lebesgue-mitallinen} \right\}.$$
    (Lehrbäck, MII, s. 27)
\end{definition}

\begin{definition}
    \textbf{Lebesguen mitta} Funktio $m : \mathcal M \to \R^+$ on \textit{Lebesguen mitta} kun määritellään
    $$m(A) = m^*(A) \text{ kaikille } A\subset \mathcal M.$$
\end{definition}

\subsection{HATARALLA POHJALLA:}
\begin{definition}
L-ääretön, L-infinity:
    \[L^\infty (A) = \{f:A\to X ;  ||f||_\infty < \infty\}\] 
    \[||f||_\infty = \esssup_{x\in A} ||f(x)||_{\R^n}\]
\end{definition}



\begin{definition}
    todennäköisyysmitta
\end{definition}


\begin{definition}
    Caratheodoryn funktio
\end{definition}





\begin{definition}
    heikko suppeneminen
\end{definition}


\begin{definition}
    \textbf{metrinen avaruus} Olkoon $M$ joukko ja $d$ etäisyys joukossa $M$. Tällöin pari $(M, d)$ on \textit{metrinen avaruus}, merkitään $M = (M, d)$.
\end{definition}

\begin{definition}
    \textbf{peite, osapeite} Olkoon $I$ indeksijoukko. Avointen joukkojen kokoelma $\{U_i \colon i \in I\}$ metrisessä avaruudessa $M$ on 
\end{definition}

\begin{definition}
    \textbf{kompakti metrinen avaruus} Metrinen avaruus $M$ on \textit{kompakti}, jos jokaisella avaruuden $M$ avoimella peitteellä on äärellinen osapeite.
\end{definition}
\begin{definition}
    todennäköisyysavaruus
\end{definition}


\begin{definition}
    esikompakti
\end{definition}

\begin{theorem}
    Skorokhodin lause: Olkoon (K, d) esikompakti metrinen avaruus ja $\mu$ todennäköisyysmitta joukossa $K$. Tällöin on olemassa mitta $\chi$ jolla  $\mu$ voidaan kirjoittaa muodossa
    $$ \mu = \chi_\# \lambda. $$
    Sanotaan, että $\chi$ on mitan $\mu$ parametrisaatio.
\end{theorem}
\begin{proof}
Bernot, Caselles: Optimal transportation networks s.185
\end{proof}

\end{document}