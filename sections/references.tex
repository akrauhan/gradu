\begin{document}
%%%%%%%%%%%%%%%%%%%%%%%%%%%%%%%%%%%%%%%%%%%%%%%%%
\appendix
%%%%%%%%%%%%%%%%%%%%%%%%%%%%%%%%%%%%%%%%%%%%%%%%
\backmatter
%\renewcommand{\bibname}{Lähdeluettelo}
\begin{thebibliography}{30}
\selectlanguage{english}

%\bibitem{Concrete}
%\textsc{Ronald L. Graham}, \textsc{Donald E. Knuth} ja \textsc{Oren Patashnik}:
%\textit{Concrete Mathematics. A Foundation for Computer Science}.
%toinen laitos, Addison-Wesley, 1994.

\bibitem{OptimalTransportationNetworks}
\textsc{Marc Bernot}, \textsc{Vicent Caselles} ja \textsc{Jean-Michel Morel}:
\textit{Optimal Transportation Networks. Models and Theory.}
Springer-Verlag Berlin Heidelberg. 2009.

\bibitem{MeasureTheoryTao}
\textsc{Terence Tao}:
\textit{An introduction to measure theory}.
American Mathematical Society; Uudistettu painos. 2011. \\
\url{https://terrytao.files.wordpress.com/2012/12/gsm-126-tao5-measure-book.pdf}
\selectlanguage{finnish}

\bibitem{principles}
\textsc{Walter Rudin}:
\textit{Principles of Mathematical Analysis.}
McGraw-Hill Education; 3. painos. 1976.

%\bibitem{Johdatus}
%\textsc{Ernst Lindel\"of}:
%\textit{Johdatus korkeampaan analyysiin}.
%toinen, korjattu laitos, Mercatorin Kirjapaino Osakeyhti\"o, 1926.

\end{thebibliography}
%%%%%%%%%%%%%%%%%%%%%%%%%%%%%%%%%%%%%%%%%%%%%%%%
\end{document}