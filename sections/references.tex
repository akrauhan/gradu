\begin{document}
%%%%%%%%%%%%%%%%%%%%%%%%%%%%%%%%%%%%%%%%%%%%%%%%%
\appendix
%%%%%%%%%%%%%%%%%%%%%%%%%%%%%%%%%%%%%%%%%%%%%%%%
\backmatter
%\renewcommand{\bibname}{Lähdeluettelo}
\begin{thebibliography}{30}
\selectlanguage{english}

%\bibitem{Concrete}
%\textsc{Ronald L. Graham}, \textsc{Donald E. Knuth} ja \textsc{Oren Patashnik}:
%\textit{Concrete Mathematics. A Foundation for Computer Science}.
%toinen laitos, Addison-Wesley, 1994.

\bibitem{optimal}
\textsc{Marc Bernot}, \textsc{Vicent Caselles} ja \textsc{Jean-Michel Morel}:
\textit{Optimal Transportation Networks. Models and Theory.}
Springer-Verlag Berlin Heidelberg. 2009.

\bibitem{tao}
\textsc{Terence Tao}:
\textit{An introduction to measure theory}.
American Mathematical Society; Uudistettu painos. 2011.
\selectlanguage{finnish}

\bibitem{rudin}
\textsc{Walter Rudin}:
\textit{Principles of Mathematical Analysis.}
McGraw-Hill Education; 3. painos. 1976.

\bibitem{lehrbäck}
\textsc{Juha Lehrbäck}:
\textit{Mitta- ja Integraaliteoria (Osat 1 ja 2).}
{Jyväskylän yliopisto.} 2018.

\bibitem{evans}
\textsc{Lawrcence C. Evans}:
\textit{Partial Differential Equations and Monge-Kantorovich Mass Transfer.}
{University of California, Berkeley.} 2001.
\url{https://math.berkeley.edu/~evans/Monge-Kantorovich.survey.pdf}

\bibitem{monge}
\textsc{G. Monge.}:
\textit{Mémoire sur la théorie des déblais et de remblais. Histoire de l’Académie Royale des Sciences de Paris, avec les Mémoires de Mathématique et de Physique pour la méme année}. s.666-704. 1781.

\bibitem{bogachev}
\textsc{Vladimir I. Bogachev}:
\textit{Measure Theory, Nide 1.}
{Springer Science ja Business Media.} 2007.

\bibitem{burago}
\textsc{Dimitri Burago, Yuri Burago, Sergei Ivanov}:
\textit{A Course in Metric Geometry.}
{American Mathematical Society.} 2001.

\bibitem{conway}
\textsc{John B. Conway}:
\textit{A Course in Functional Analysis.}
{Springer-Verlag New York.} 1985.

\bibitem{benesova}
\textsc{Barbora Benešová, Martin Kružík}
\textit{Weak lower semicontinuity of integral functionals and applications}
{Society for Industrial and Applied Mathematics.} 2017. 
\url{https://doi.org/10.1137/16M1060947}

%\bibitem{Johdatus}
%\textsc{Ernst Lindel\"of}:
%\textit{Johdatus korkeampaan analyysiin}.
%toinen, korjattu laitos, Mercatorin Kirjapaino Osakeyhti\"o, 1926.

\end{thebibliography}
%%%%%%%%%%%%%%%%%%%%%%%%%%%%%%%%%%%%%%%%%%%%%%%%
\end{document}